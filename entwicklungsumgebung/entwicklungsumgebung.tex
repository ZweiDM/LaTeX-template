%
%%%%%%%%%%%%%%%%%%%%%%%%%%%%%%%%%%%%%%%%%%%%%%%%%%%
%
%  E N T W I C K L U N G S U M G E B U N G
%
%%%%%%%%%%%%%%%%%%%%%%%%%%%%%%%%%%%%%%%%%%%%%%%%%%%
\chapter{Entwicklungsumgebung}
\label{cha:umgebung}
%
%
Als Entwicklungsumgebung f�r die Erstellung und Bearbeitung von Latex-Dokumenten unter Windows eignet sich am besten die Kombination aus MiKTeX\footnote{www.miktex.org} und TeXstudio\footnote{texstudio.sourceforge.net/}. Auf heise.de\footnote{www.heise.de/software} kann man auch drekt das TeXstudio als Portable-Version herunterladen, die nicht einmal installiert werden muss und auf jedem USB-Stick Platz findet.\\

Laden und installieren Sie zun�chst MiKTeX, dann installieren oder �ffnen Sie das TeXstudio. Laden Sie die Haupt-.tex-Datei der Abschlussarbeit. Auf der linken Seite sollten nun auch alle Einzelkapitel, Anh�nge und das Glossar angezeigt werden. Durch dr�cken des gr�nen Doppelpfleils in der Icon-Leiste wird das fertige .pdf erstellt und direkt im Studio angezeigt. Au�erdem liegt das fertige .pdf nat�rlich auch im Projektordner.\\

Zum Bearbeiten eines Kapitels Ihrer Abschlussarbeit dr�cken Sie einfach im linken Struktur-Fenster auf das entsprechende Kapitel und editieren Sie den Inhalt.\\ 
